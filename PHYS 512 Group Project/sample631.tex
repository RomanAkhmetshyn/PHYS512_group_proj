%% Beginning of file 'sample631.tex'
%%
%% Modified 2022 May  
%%
%% This is a sample manuscript marked up using the
%% AASTeX v6.31 LaTeX 2e macros.
%%
%% AASTeX is now based on Alexey Vikhlinin's emulateapj.cls 
%% (Copyright 2000-2015).  See the classfile for details.

%% AASTeX requires revtex4-1.cls and other external packages such as
%% latexsym, graphicx, amssymb, longtable, and epsf.  Note that as of 
%% Oct 2020, APS now uses revtex4.2e for its journals but remember that 
%% AASTeX v6+ still uses v4.1. All of these external packages should 
%% already be present in the modern TeX distributions but not always.
%% For example, revtex4.1 seems to be missing in the linux version of
%% TexLive 2020. One should be able to get all packages from www.ctan.org.
%% In particular, revtex v4.1 can be found at 
%% https://www.ctan.org/pkg/revtex4-1.

%% The first piece of markup in an AASTeX v6.x document is the \documentclass
%% command. LaTeX will ignore any data that comes before this command. The 
%% documentclass can take an optional argument to modify the output style.
%% The command below calls the preprint style which will produce a tightly 
%% typeset, one-column, single-spaced document.  It is the default and thus
%% does not need to be explicitly stated.
%%
%% using aastex version 6.3
\documentclass[linenumbers]{aastex631}

%% The default is a single spaced, 10 point font, single spaced article.
%% There are 5 other style options available via an optional argument. They
%% can be invoked like this:
%%
%% \documentclass[arguments]{aastex631}
%% 
%% where the layout options are:
%%
%%  twocolumn   : two text columns, 10 point font, single spaced article.
%%                This is the most compact and represent the final published
%%                derived PDF copy of the accepted manuscript from the publisher
%%  manuscript  : one text column, 12 point font, double spaced article.
%%  preprint    : one text column, 12 point font, single spaced article.  
%%  preprint2   : two text columns, 12 point font, single spaced article.
%%  modern      : a stylish, single text column, 12 point font, article with
%% 		  wider left and right margins. This uses the Daniel
%% 		  Foreman-Mackey and David Hogg design.
%%  RNAAS       : Supresses an abstract. Originally for RNAAS manuscripts 
%%                but now that abstracts are required this is obsolete for
%%                AAS Journals. Authors might need it for other reasons. DO NOT
%%                use \begin{abstract} and \end{abstract} with this style.
%%
%% Note that you can submit to the AAS Journals in any of these 6 styles.
%%
%% There are other optional arguments one can invoke to allow other stylistic
%% actions. The available options are:
%%
%%   astrosymb    : Loads Astrosymb font and define \astrocommands. 
%%   tighten      : Makes baselineskip slightly smaller, only works with 
%%                  the twocolumn substyle.
%%   times        : uses times font instead of the default
%%   linenumbers  : turn on lineno package.
%%   trackchanges : required to see the revision mark up and print its output
%%   longauthor   : Do not use the more compressed footnote style (default) for 
%%                  the author/collaboration/affiliations. Instead print all
%%                  affiliation information after each name. Creates a much 
%%                  longer author list but may be desirable for short 
%%                  author papers.
%% twocolappendix : make 2 column appendix.
%%   anonymous    : Do not show the authors, affiliations and acknowledgments 
%%                  for dual anonymous review.
%%
%% these can be used in any combination, e.g.
%%
%% \documentclass[twocolumn,linenumbers,trackchanges]{aastex631}
%%
%% AASTeX v6.* now includes \hyperref support. While we have built in specific
%% defaults into the classfile you can manually override them with the
%% \hypersetup command. For example,
%%
%% \hypersetup{linkcolor=red,citecolor=green,filecolor=cyan,urlcolor=magenta}
%%
%% will change the color of the internal links to red, the links to the
%% bibliography to green, the file links to cyan, and the external links to
%% magenta. Additional information on \hyperref options can be found here:
%% https://www.tug.org/applications/hyperref/manual.html#x1-40003
%%
%% Note that in v6.3 "bookmarks" has been changed to "true" in hyperref
%% to improve the accessibility of the compiled pdf file.
%%
%% If you want to create your own macros, you can do so
%% using \newcommand. Your macros should appear before
%% the \begin{document} command.
%%
\newcommand{\vdag}{(v)^\dagger}
\newcommand\aastex{AAS\TeX}
\newcommand\latex{La\TeX}

%% Reintroduced the \received and \accepted commands from AASTeX v5.2
%\received{March 1, 2021}
%\revised{April 1, 2021}
%\accepted{\today}

%% Command to document which AAS Journal the manuscript was submitted to.
%% Adds "Submitted to " the argument.
%\submitjournal{PSJ}

%% For manuscript that include authors in collaborations, AASTeX v6.31
%% builds on the \collaboration command to allow greater freedom to 
%% keep the traditional author+affiliation information but only show
%% subsets. The \collaboration command now must appear AFTER the group
%% of authors in the collaboration and it takes TWO arguments. The last
%% is still the collaboration identifier. The text given in this
%% argument is what will be shown in the manuscript. The first argument
%% is the number of author above the \collaboration command to show with
%% the collaboration text. If there are authors that are not part of any
%% collaboration the \nocollaboration command is used. This command takes
%% one argument which is also the number of authors above to show. A
%% dashed line is shown to indicate no collaboration. This example manuscript
%% shows how these commands work to display specific set of authors 
%% on the front page.
%%
%% For manuscript without any need to use \collaboration the 
%% \AuthorCollaborationLimit command from v6.2 can still be used to 
%% show a subset of authors.
%
%\AuthorCollaborationLimit=2
%
%% will only show Schwarz & Muench on the front page of the manuscript
%% (assuming the \collaboration and \nocollaboration commands are
%% commented out).
%%
%% Note that all of the author will be shown in the published article.
%% This feature is meant to be used prior to acceptance to make the
%% front end of a long author article more manageable. Please do not use
%% this functionality for manuscripts with less than 20 authors. Conversely,
%% please do use this when the number of authors exceeds 40.
%%
%% Use \allauthors at the manuscript end to show the full author list.
%% This command should only be used with \AuthorCollaborationLimit is used.

%% The following command can be used to set the latex table counters.  It
%% is needed in this document because it uses a mix of latex tabular and
%% AASTeX deluxetables.  In general it should not be needed.
%\setcounter{table}{1}

%%%%%%%%%%%%%%%%%%%%%%%%%%%%%%%%%%%%%%%%%%%%%%%%%%%%%%%%%%%%%%%%%%%%%%%%%%%%%%%%
%%
%% The following section outlines numerous optional output that
%% can be displayed in the front matter or as running meta-data.
%%
%% If you wish, you may supply running head information, although
%% this information may be modified by the editorial offices.
%\shorttitle{AASTeX v6.3.1 Sample article}
%\shortauthors{Schwarz et al.}
%%
%% You can add a light gray and diagonal water-mark to the first page 
%% with this command:
%% \watermark{text}
%% where "text", e.g. DRAFT, is the text to appear.  If the text is 
%% long you can control the water-mark size with:
%% \setwatermarkfontsize{dimension}
%% where dimension is any recognized LaTeX dimension, e.g. pt, in, etc.
%%
%%%%%%%%%%%%%%%%%%%%%%%%%%%%%%%%%%%%%%%%%%%%%%%%%%%%%%%%%%%%%%%%%%%%%%%%%%%%%%%%
%\graphicspath{{./}{figures/}}
%% This is the end of the preamble.  Indicate the beginning of the
%% manuscript itself with \begin{document}.

\begin{document}

\title{PHYS 512 Project: Transit Timing Variation}

%% LaTeX will automatically break titles if they run longer than
%% one line. However, you may use \\ to force a line break if
%% you desire. In v6.31 you can include a footnote in the title.

%% A significant change from earlier AASTEX versions is in the structure for 
%% calling author and affiliations. The change was necessary to implement 
%% auto-indexing of affiliations which prior was a manual process that could 
%% easily be tedious in large author manuscripts.
%%
%% The \author command is the same as before except it now takes an optional
%% argument which is the 16 digit ORCID. The syntax is:
%% \author[xxxx-xxxx-xxxx-xxxx]{Author Name}
%%
%% This will hyperlink the author name to the author's ORCID page. Note that
%% during compilation, LaTeX will do some limited checking of the format of
%% the ID to make sure it is valid. If the "orcid-ID.png" image file is 
%% present or in the LaTeX pathway, the OrcID icon will appear next to
%% the authors name.
%%
%% Use \affiliation for affiliation information. The old \affil is now aliased
%% to \affiliation. AASTeX v6.31 will automatically index these in the header.
%% When a duplicate is found its index will be the same as its previous entry.
%%
%% Note that \altaffilmark and \altaffiltext have been removed and thus 
%% can not be used to document secondary affiliations. If they are used latex
%% will issue a specific error message and quit. Please use multiple 
%% \affiliation calls for to document more than one affiliation.
%%
%% The new \altaffiliation can be used to indicate some secondary information
%% such as fellowships. This command produces a non-numeric footnote that is
%% set away from the numeric \affiliation footnotes.  NOTE that if an
%% \altaffiliation command is used it must come BEFORE the \affiliation call,
%% right after the \author command, in order to place the footnotes in
%% the proper location.
%%
%% Use \email to set provide email addresses. Each \email will appear on its
%% own line so you can put multiple email address in one \email call. A new
%% \correspondingauthor command is available in V6.31 to identify the
%% corresponding author of the manuscript. It is the author's responsibility
%% to make sure this name is also in the author list.
%%
%% While authors can be grouped inside the same \author and \affiliation
%% commands it is better to have a single author for each. This allows for
%% one to exploit all the new benefits and should make book-keeping easier.
%%
%% If done correctly the peer review system will be able to
%% automatically put the author and affiliation information from the manuscript
%% and save the corresponding author the trouble of entering it by hand.

%\correspondingauthor{August Muench}
%\email{greg.schwarz@aas.org, gus.muench@aas.org}

\author{Roman Akhmetshyn}

\author{Sarah Silverman}

\author{Steven Hsueh}

\keywords{Transit Timing Variation --- Exoplanetary Astronomy --- Limb Darkening --- MCMC --- N-body simulations}

%% From the front matter, we move on to the body of the paper.
%% Sections are demarcated by \section and \subsection, respectively.
%% Observe the use of the LaTeX \label
%% command after the \subsection to give a symbolic KEY to the
%% subsection for cross-referencing in a \ref command.
%% You can use LaTeX's \ref and \label commands to keep track of
%% cross-references to sections, equations, tables, and figures.
%% That way, if you change the order of any elements, LaTeX will
%% automatically renumber them.
%%
%% We recommend that authors also use the natbib \citep
%% and \citet commands to identify citations.  The citations are
%% tied to the reference list via symbolic KEYs. The KEY corresponds
%% to the KEY in the \bibitem in the reference list below. 

\section{Introduction} \label{sec:intro}

The transit method has revolutionized exoplanetary science: of the 5,788 confirmed exoplanets discovered to date, 4,310 have been discovered using this observational technique \citep{exoplanetcandidates}. When an exoplanet passes between an observer (us on Earth) and its host star, the star's light is temporarily dimmed. The transit method measures the dip in the star’s light curve caused by that dimming. Remarkably, by analyzing these transit light curves, we can ascertain many of the planet’s characteristics, such as size (i.e., radius), orbital period, and even atmospheric composition (with multi-wavelength transit observations). \par \vspace{.1 cm}
However, roughly $\frac{1}{5}$ of the exoplanets discovered to date are in multi-planet systems, where planets co-orbiting the star interact gravitationally and potentially perturb their respective rotational and orbital rates \citep{exoplanetcandidates}. This interaction may nudge planets out of a periodic orbit, causing a \textbf{Transit Timing Variation} (TTV). The TTV observational method is advantageous and robust because it exploits this multi-planetary interaction to constrain the parameters listed for the single-planet case and obtain planetary mass. This is because the amplitude of the timing variations depends on the mass ratio between the two planets. Larger perturbing planets have a more significant gravitational influence and, therefore, cause a more pronounced timing variation. TTVs are especially relevant in the context of planets in mean-motion resonance—i.e., planets whose orbital periods have an integer ratio (e.g., 2:1, 3:2)—because such planets exert perturbations on one another that are periodic and, hence, predictable \citep{meanmotionresonance}. 
\par \vspace{.1 cm}
In both the single-planet and multi-planet scenarios, an important consideration when simulating and observing transit light curves is the effect called \textbf{limb darkening}, which we account for in our analysis (see section \ref{sec:Methods}). Stellar limb darkening is a phenomenon in which the edge of a star appears darker than the center of the disk. This effect is essential to consider during exoplanet transits because when the planet passes in front of the darker “limb” at the beginning of the transit, the star's brightness decreases more gradually than if the star were uniformly bright. This leads to a smoother initial dip in the light curve, sharpening as the planet passes in front of the star's center. Accounting for limb darkening is crucial because it ensures that we correctly estimate how much light the planet is blocking at different parts of the star, which impacts the shape of the transit light curve and, hence, the properties inferred about the transiting planet.
 \par \vspace{.1 cm}
Our project is split between three objectives. In Part I, we create an N-body simulation of 2-planet system to demonstrate the effect of gravitational interactions that causes the transit time variation. In Part II, we consider the transit of a single planet around its host star and use the Markov Chain Monte Carlo (MCMC) to constrain the best-fit orbital period and planetary radius parameters. Part II extends this analysis to consider the TTV resulting from a two-planet scenario by using MCMC again along with the N-body simulation developed in Part I.
\vspace{1cm}
\begin{table}[h!]
    \centering
    \begin{tabular}{||c|c|c||}
         \hline
         Parameter & Meaning & Units \\ [0.5ex] 
         \hline\hline 
         *eccentricity (e) & Describes shape of the orbit & 0$\leq$e$<$1  \\
         \hline 
         semi-major axis (a) & Describes size of the orbit & R$_{star}$ \\
         \hline
         $^1$inclination (i) & The tilt of the orbital plane (90 = edge on orbit) & Degrees  \\
         \hline
         %Longitude of the Ascending Node ($\Omega$) & The orientation of the orbital plane in 3D space. & $^{\circ}$ \\
         %\hline
         *Argument of Periapsis ($\omega$) & The orientation of the ellipse within the orbital plane. & $^{\circ}$ \\
         \hline
         %Longtiude of the Periapis ($\varpi$) & $\Omega$ + $\omega$ & $^{\circ}$ \\
         %\hline
         *True Anomaly ($\nu$) & The position of the object along the orbit at a given time. & $^{\circ}$ \\
         \hline
         Orbital Period (T) & Time of one orbit & Days \\ 
         \hline
         $t_{0}$ & Time that the planet reaches the mid-transit point & Days\\
         \hline
         $R_{s}$ & Planet radius & R$_{star}$  \\
         \hline
         $^2$limb darkening law & equation that characterizes limb darkening & n/a \\
         \hline
         limb darkening coefficients & $u_1$ and $u_2$& n/a \\ [1ex] 
         \hline
    \end{tabular}
    \caption{Parameters of orbital dynamics \textbf{obtainable by transit observations and TTV analysis}.\\ $^*$Parameters that are hard to constrain for both planets without other methods involved (i. e. Radial Velocity observations) or some bold assumptions. \\ $^1$Often assumed to be $\approx90^o$ for transiting planets. \\ $^2$We use quadratic law: $I(r)=1 - u_1(1 - \mu) - u_2(1 - \mu)^2$.}
    \label{observed_params}
\end{table}
\vspace{1cm}
\newpage
\section{METHODS} \label{sec:Methods}
\subsection{Part I: N-body simulation of a transit timing variation}
i. {\tt\string Leapfrog} \label{sec:Leapfrog}
\par \vspace{.01cm}
In this part, we simulate a two-planet star system and explore how the planets interact with each other to produce the effect of TTVs. This simulation uses simple gravitational interactions between three bodies: the star, planet 1, and planet 2. Specifically, we observe the transit of planet 1, with its timing variations caused by the gravitational influence of planet 2. This code models the motion of the planets and demonstrates how planet-planet interactions lead to variations in transit times.

To implement this N-body simulation, we use the {\tt\string leapfrog} method – a simple technique with good second-order accuracy and energy conservation. This method is similar to a basic (lazy) approach for calculating two-body gravitational interactions: at each time step, we calculate the force exerted based on the distance between two bodies, update the positions using the previous velocity, and then update the velocity based on the acceleration derived from the force. However, this straightforward approach is unsuitable for modeling planetary systems because it does not conserve energy and leads to cumulative errors and numerical instabilities over time. One improvement involves averaging positions and accelerations across two-time steps to update the system, but an even better method is to reorder the computation steps.

In the {\tt\string leapfrog} method, positions are updated at full time steps, while velocities are updated at half time steps. This reordering means we update positions using velocities evaluated at the midpoint of the time step, while forces are computed based on the next position. This achieves second-order accuracy and better conserves energy.

Our simulation involves a large loop with extensive calculations. To optimize performance, we use the {\tt\string Numba} just-in-time compiler, which significantly speeds up the code by precompiling key functions.

First, we demonstrate the core functions of the simulation and verify that energy is conserved. Then, we compare computation times between the standard implementation and the {\tt\string Numba}-optimized version.

Given the coplanar nature of our leapfrog simulation, we make the following simplifications about parameters in Table ~\ref{observed_params}: i and $\Omega$ are irrelevant since there is only a single rotational axis for the orbital plane; $\omega$ defines the orientation of the orbital plane; we assume the objects start at periapsis, setting $\nu$ to zero.
\subsection{Part II: MCMC modelling of the single planet transit}

i. {\tt\string batman}
\par \vspace{.1cm}
Given the increasing rate at which exoplanets are discovered yearly \citep{ourworldindata2024}, developing reliable light curve models is essential for interpreting observational data and extracting properties about the planets and their host stars. With this in mind, Professor Laura Kriedberg at the University of Chicago introduced a Python package {\tt\string batman}: \textbf{BAsic Transit Model cAlculatioN}, which is designed to expedite light curve computations while maintaining their accuracy \citep{batman}. Importantly, this package considers the role of limb darkening and can be applied to any radially symmetric limb darkening laws, including uniform, linear, quadratic, logarithmic, exponential, and nonlinear.
\par \vspace{.1 cm}
Underscoring the importance of code performance as discussed in class, {\tt\string batman} uses C extension modules to speed up computation time by “a factor of 30 over pure Python implementation for quadratic limb darkening.” Moreover, by using OpenMP, the {\tt\string batman} code is efficiently improved with parallelization.
\par \vspace{.1 cm}
Our work with CAMB in Homework 4 and its role when fitting Cosmic Microwave Background parameters with MCMC influenced our choice to use {\tt\string batman} for our MCMC. {\tt\string batman} required nine input parameters to generate the model light curve: time of the mid-transit point ($t_{0}$), orbital period (T), planet radius in units of stellar radii ($R_{p}$/$R_{s}$), semi-major axis in units of stellar radii ($a/R_{s}$), orbital inclination (\textit{i}), eccentricity (\textit{e}), longitude of the periapsis ($\varpi$), limb darkening coefficients, and the limb darkening law. In exoplanetary science, inclination refers to the angle between the orbital plane of the planet and the plane of the sky as seen from Earth. Exoplanet transits can only be observed when the planet’s inclination is close to 90$^{\circ}$. Many planets in evolved systems, such as those in the TRAPPIST-1 system, have stabilized inclinations around this value. Consequently, in our simulations, we simplified the problem by considering motion in a single plane and fix \textit{i} as 90$^{\circ}$. We also fixed $\varpi$ as 90$^{\circ}$ and \textit{e} as 0$^{\circ}$. Lasty, we chose “quadratic” as our limb-darkening model. We aimed to find the best-fit parameters for T, $R_{p}$/$R_{s}$, $a/R_{s}$, and the limb darkening coefficients, so that we could best characterize the nature of the transiting planet. 
\par \vspace{.1cm}
ii. {\tt\string starry}
\par \vspace{.1cm}
To run MCMC, we also needed “true” data, \textit{i.e.}, a transit light curve from an observation. We simulated a true observation by generating a light curve using the python package {\tt\string starry} \citep{starry} and injecting Gaussian noise into the flux. We note that {\tt\string starry} and {\tt\string batman} are actually equivalent for transit modeling, but we were interested in being acquainted with multiple different exoplanet transit models and so used both in our project. In the case of {\tt\string starry}, however, noise is added to the simulated light curve.


%\begin{table}
%    \centering
%    \begin{tabular}{||c|c||}
%         \hline
%         Parameter & Observed \\ [0.5ex] 
 %        \hline\hline 
 %        t0 (hours) & 10/24  \\
 %        \hline 
 %        T (days) & 10  \\
%         \hline
%         $R_{p}/R_{s}$ & .1  \\
%         \hline
%         $a/R_{s}$ & 19.5177  \\
%         \hline
%         i (degrees) & 90 \\
%         \hline
%        e & 0 \\
%         \hline
%         w (degrees) & 90 \\
%         \hline
%         limb darkening law & quadratic \\
%         \hline
%         limb darkening coefficients & [.4, .1] \\ [1ex] 
%         \hline
%    \end{tabular}
%    \caption{Parameters for an observed transit light curve. \textit{Note}: a is the semi-major axis; $R_{p}$ and $R_{s}$ are the planet and star radii, respectively.}
%    \label{tab:observed_params}
%\end{table}



\subsection{Part III: MCMC modeling of a transit timing variation}
\par \vspace{.01cm}
To estimate more parameters about our observed transiting exoplanet and to infer some of the new parameters of the second body (perturber), similarly to the previous part, we will utilize MCMC. In this case, however, we will rely completely on the N-body model we simulated in Part I. 
\par \vspace{.1 cm}
From Part II, we find out that the mid-transit point ($t_{0}$) of each subsequent transit is not increasing periodically. In other words, we experience TTV and the period of the exoplanet is not constant, indicating a perturbation. In TTV research a common indicator of these perturbations is the Observed-Calculated (O-C) curve: the difference between the observed mid-transit point and the estimated mid-transit point from the period. In our case, the calculated period is the average period taken from all timing observations. The shape, magnitude, period, and slope of the curve can tell us a lot about the number of perturbers and their orbital and physical parameters. Thus, we perform MCMC sampling on the O-C curve with a given set of prior guesses.
\par \vspace{.1 cm}
As shown in Part I, the N-body simulation function takes a total of 10 parameters, and that's only if we consider 2D orbits. In terms of the O-C curve, this can be a degenerate problem and lead to correlations between some parameters. We restructure our prior parameters to reduce degeneracy while maintaining a physical sense. We assume that the mass of the star (M$_{star}$) is equal to 1 solar mass, let's just say that previous spectroscopic observations in combination with stellar evolution models and the fact that the star is in the stellar association provided us with age constraints. Instead of sampling the masses of both planets, we sample the mass of the second as a ratio of $M_1/M_2$. We assign the initial period of the first planet to 10 days, which we got from quasi-periodic transit observations. The eccentricity and the argument of periapsis are left to be equal to 0 from previous assumptions and from the fact that it's impossible to determine without the radial velocity method. 



\section{Results} \label{sec:results}
All the results and discussions are demonstrated in \texttt{Jupyter Notebooks} as follows:
\begin{enumerate}
    \item Part I: \texttt{N-body part.ipynb}
    \item Part II: \texttt{Transit part.ipynb}
    \item Part III: \texttt{TTV part.ipynb}
\end{enumerate}

\bibliography{sample631}{}
\bibliographystyle{aasjournal}

%% This command is needed to show the entire author+affiliation list when
%% the collaboration and author truncation commands are used.  It has to
%% go at the end of the manuscript.
%\allauthors

%% Include this line if you are using the \added, \replaced, \deleted
%% commands to see a summary list of all changes at the end of the article.
%\listofchanges

\end{document}

% End of file `sample631.tex'.
